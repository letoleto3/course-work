\documentclass[12pt]{extarticle}
\usepackage[utf8]{inputenc}
\usepackage[english,russian]{babel}
\usepackage{vmargin}
\usepackage{indentfirst}
\usepackage[T2A]{fontenc}
\usepackage{graphics}
\usepackage{amsthm}
\usepackage{amsbsy}
\usepackage{amsmath}
\usepackage{amssymb}
\usepackage{amsfonts}
\usepackage{mathtext}
\usepackage[pdftex,a4paper,colorlinks,linkcolor=blue,citecolor=blue]{hyperref}	


\usepackage[pdftex]{graphicx}
\usepackage{array}
\usepackage{graphicx,xcolor}
\usepackage{xcolor}
\usepackage{float}
\usepackage{longtable}

\usepackage[]{color}

\parindent = 30pt
\hoffset = 0pt
\voffset = 0pt
\oddsidemargin = 72pt
\topmargin = 18pt
\headheight = 12pt
\headsep = 25pt
\textheight = 568pt
\textwidth = 360pt
\marginparsep = 10pt
\marginparwidth = 103pt

\usepackage{tikz}
\usepackage{verbatim}

\paperwidth = 597pt
\paperheight = 845pt

\begin{document}

\begin{titlepage} \newpage 
\begin{center} МОСКОВСКИЙ ГОСУДАРСТВЕННЫЙ УНИВЕРСИТЕТ\\ ИМ. М.В. ЛОМОНОСОВА\end{center} 
\vspace{8em} \begin{center} 
\Large Механико-математический факультет \\ \end{center}
\vspace{2em} \begin{center} 
\textsc{Курсовая работа \linebreak \textbf{}
\linebreak \linebreak \textbf{}} \end{center}
\vspace{6em} \newbox{\lbox} \savebox{\lbox}{\hbox{Я.Нагорных}} 
\newlength{\maxl} \setlength{\maxl}{\wd\lbox} \hfill\parbox{12	cm}
{ \hspace*{10cm}\hspace*{-5cm}Студент 3 курса:\hfill\hbox to\maxl{Нагорных Я.В.}\\
\hspace*{10cm}\hspace*{-5cm}Научный руководитель:\hfill\hbox to\maxl{Богачев К.Ю.}}\\  \vspace{\fill}
\begin{center} Москва \\ 2017\end{center} \end{titlepage}

%\maketitle

%\renewcommand{\contentsname}{Содержание}
\tableofcontents

\renewcommand{\figurename}{Рисунок}

\newpage

\section*{Введение}
\addcontentsline{toc}{section}{Введение}
Печать большив массивов чисел всегда занимает много времени.
Кроме того, при печати мало ресурсов для ускорения.

Печать чисел с плавающей запятой также является была проблемой. 
Стандартный подход недостаточно точен и в некоторых случаях дает неверные результаты. 
Кроме того использование функций стандартных библиотек (\texttt{printf}, \texttt{sprintf}) достаточно затратно по времени.
\\
\\
\textbf{Цели работы:}
\begin{enumerate}
\item Ускорить печать больших массивов;
\item Использовать быстрые алгоритмы печати целых чисел и чисел с плавающей точкой.
\end{enumerate}

\section{Проблемы и способы их решения}
Как уже было сказано, при печати массивов мало ресурсов для ускорения.
Также проблемой является и то, что печать данных файл должна быть строго последовательной, поэтому нельзя ''простым'' образом использовать распараллеливание.

Однако, известно что большую часть времени занимает преобразование типа \texttt{int} или \texttt{double} в буффер типа \texttt{const char *} непосредственно для печати.
Именно это можно и распараллелить, используя многопоточное программирование.
Непосредственно печать в сам файл упирается в возможности диска. 
Ее ускорить нельзя.

Кроме того, можно заменить стандартный алгоритм преобразования числа в строку, на более быстрые.
Мы будем использовать алгоритм \textsf{Grisu2}, о котором будет рассказано позже.


\section{Описание алгоритма}
В классе   \texttt{\textcolor[rgb]{0.5,0,0.5}{parallel\_writer}}

\subsection{Распределение задач}


\subsection{Описание алгоритма}
\section{Заключение}


\newpage

\appendix
\section*{Приложение}
\addcontentsline{toc}{section}{Приложение}
\renewcommand{\thesubsection}{(\Alph{subsection})}

\newpage
\begin{thebibliography}{}

\bibitem{1} \textsc{Florian Loitsch}, 
Printing Floating-Point Numbers Quickly and Accurately with Integers, 2004.
\bibitem{2} \textsc{Богачев К. Ю.}, 
Основы параллельного программирования. -- M.: Бином. Лаборатория знаний, 2010.


\end{thebibliography}

\end{document}
