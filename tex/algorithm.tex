\section{Описание алгоритма} \label{sec1}
Схематично работа алгоритма параллельной записи показана на Рисунке \ref{draw}.
\begin{figure}[h!]
\begin{footnotesize}
\def\svgwidth{430pt}
  \input{pics/drawing.pdf_tex}
  \caption{Работа потоков.} \label{draw}
\end{footnotesize}\end{figure}

\subsection{Используемые структуры и классы}
\paragraph{Структура \texttt{\class{reduce\_chunk}}.}
В ней находится строковый буфер, готовый для печати, и его порядковый номер \texttt{\field{chunk\_id}}.
Кроме того, хранится флаг, является ли этот \texttt{\class{reduce\_chunk}} последним.
\paragraph{Структура \texttt{\class{map\_chunk}}.}
Этот тип состоит из лямбда-функции, которая должна обработать определенный фрагмент массива чисел, и элемента типа \texttt{\class{reduce\_chunk}}, возвращаемый функцией.
\paragraph{Класс \texttt{\class{mutex\_wait\_queue}}.}
Это упрощенная реализация \textit{очереди сообщений}.  
Под ней понимается очередь со следующим свойством: когда поток пытается прочитать что-то из пустой очереди, то он блокируется, до тех пор, пока какой-нибудь другой поток не положит в нее элемент.
У этой очереди есть следующие методы:
\begin{itemize}
\item \texttt{dequeue} -- достает верхний элемент из очереди, если очередь непустая.
Иначе, поток, вызвавший этот метод блокируется. Также можно передать время блокировки, по истечении которого, поток разблокируется и вернется ни с чем;
\item \texttt{dequeue\_all} -- аналогично \texttt{dequeue}, но достает все элементы, находящиеся в очереди, и складывает в переданный указатель вектор из них;
\item \texttt{enqueue} -- кладет элемент в конец очереди.
\end{itemize}
\paragraph{Класс \texttt{\class{worker\_thread}}.}
Это главный управляющий поток.
Он хранит две очереди сообщений \texttt{\field{m\_map\_queue}} и \texttt{\field{m\_reduce\_queue}}, состоящие из \texttt{\class{map\_chunk}} и \texttt{\class{reduce\_chunk}} соответственно. 
Зачем нужны такие очереди, будет сказано позже.

\subsection{Распределение задач}
Управляющий, или главный, поток \texttt{\class{worker\_thread}} вызывает функцию \texttt{create\_mappers()}, которая создает несколько потоков-обработчиков, которые преобразовывают числа в строки, и функцию \texttt{create\_reducer()}, которая создает печатающий поток.
Сам управляющий поток будет складывать элементы типа \texttt{\class{map\_chunk}} в очередь \texttt{\field{m\_map\_queue}}.
Потоки-обработчики будут доставать из этой очереди \texttt{\class{map\_chunk}}-и на обработку и конвертировать числа в буферы типа \texttt{\class{reduce\_chunk}}, готовые для печати.
Эти готовые буферы они будут складывать в другую очередь \texttt{\field{m\_reduce\_queue}}.
Печатающий поток должен забирать все готовые буферы из этой очереди и записывать их в правильном порядке в файл.


\subsection{Преобразование чисел в строковый тип}
Мы уже говорили о том, что вместо стандартных функций \texttt{sprintf, strtod} и подобных им, будем использовать более быстрый алгоритм преобразования чисел, а именно алгоритм \textsf{Grisu2}.

Но прежде чем приступить к самому описанию этого алгоритма, сначала приведем используемые обозначения и кратко опишем его <<предшественника>> \textsf{Grisu}.  

\subsubsection{Используемые обозначения.}
Приведем используемые далее обозначения и понятия.

Мы рассматриваем стандарт \textsf{IEEE 754} -- формат представления чисел с плавающей точкой.
Знак числа хранится отдельно, поэтому далее мы будем рассматривать положительные числа, имея ввиду, что у отрицательных чисел <<->> записан в отдельном бите.

Вообще, число $v$ с плавающей точкой по основанию $b$ представляется в памяти как $$v = f_v \times b^{e_v},$$ где основание $b$ в стандарте \textsf{IEEE 754} равно $2$. $f_v$ -- целое значение или \textit{мантисса}, а $e_v$ -- \textit{показатель}.

Любая мантисса $f$ может быть представлена как $$f = \sum\limits_{i=0}^{p-1} d_i \times b^i,$$ где $0 \leqslant d_i \leqslant b$. 
Числа $d_i$ -- \textit{знаки числа}.

Будем называть число \textit{нормированным}, если последний знак $d_{p-1}$ отличен от нуля.
Если экспонента может принимать любые неограниченные значения, то любое ненулевое число можно так отнормировать путем сдвига знака влево при соответствующей корректировке экспоненты. 

В стандарте \textsf{IEEE 754} представлены не все вещественные числа. 
Из-за этого, числа, которые не могут быть представлены в этом стандарте, будем округлять.
Как известно, числа с цифрой $5$ на конце, могут округляться по-разному.
Используем следующие обозначения:
\begin{itemize}
\item $[x]^\uparrow$ -- округление вверх;
\item $[x]^\Box$ -- округление до ближайшего четного: например, число \\$0.5$ округляется до $0$, а число $1.5$ до $2$;
\item $[x]^\star$ -- используется, когда неважно, как именно округлять;
\item $\tilde x = \left[ x \right]_p^s$ -- округленное число до $p$ знаков после запятой,\\ а $s$ -- один из вышеизложенных способов округления.
\end{itemize}

Количественно определим ошибку $|\tilde x - x|$ следующим образом.
Представим $\tilde x$ в виде $\tilde x=f \times b^e$, а $f$ округлим до ближайшего такого, что $|\tilde x - x| \leqslant 0.5 \times b^e$, или другими словами, до половины единицы последнего разряда \texttt{ulp} (\textit{unit in the last place}).

Для положительного числа $v=f_v \times b^{e_v}$ определим ближайшие в памяти числа к нему.
$v^{-}$ -- предыдущее число для $v$, хранящееся в памяти.
Аналогично $v^{+}$ -- следующее число за $v$.
Если $v$ наименьшее, то $v^{-} = 0$.
Если $v$ наибольшее, то $v^{+} = v + (v - v^{-})$.

Введем понятие \textit{границы} между двумя соседними числами.\\
По определению это средние арифметические $$m^- = \cfrac{v^- + v}{2} \quad \mbox{и} \quad m^+ = \cfrac{v + v^+}{2}.$$
Очевидно, что границы нельзя представить в виде чисел с плавающей точкой, так как они лежат между двумя соседними числами в памяти. 
Поэтому любое число $w$, такое что $m^- < w < m^+$, будет округлено до $v$. 
Если же $w$ совпало с одной из границ будем округлять его до ближайшего четного.

Будем говорить, что представление $R$ у числа с плавающей точкой $v$ \textit{удовлетворяет требованию} с максимальной точностью без округления, если при чтении $R$ будет представлено как $v$.

Определим тип \texttt{diy\_fp} для $x$ как беззнаковое целое число $f_x$, состоящее из $q$ битов, и знакового целого числа $e_x$ неограниченного диапазона.
Значение $x$ можно вычислить как $x= f_x \times 2^{e_x}$. 
Другими словами, строим отображение:
$$x \mapsto (f_x, e_x), \qquad f_x \in \left(0, 2^{q-1}\right] \cap \mathbb{Z}, \quad e_x \in \mathbb{Z},\quad \mbox{т.ч.}\, x=f_x \times 2^{e_x}$$
Введем произведение двух таких типов.
Вычислять и обозначать его будем следующим образом:
$$x \otimes y := \left[ \frac{f_x \times f_y}{2^q}\right]^\uparrow \times 2^{e_x+e_y+q}.$$
\subsubsection{Алгоритм \textsf{Grisu}}
В статье \cite{1} описан алгоритм \textsf{Grisu} и его улучшения, также доказана точность его работы.
Опишем кратко этот алгоритм.
\paragraph{Идея алгоритма.}
Предполагается, не умаляя общности, что у числа с плавающей точкой $v$ отрицательный показатель. 
Тогда это число можно выразить как $$v=\cfrac{f_v}{2^{-e_v}}.$$
Десятичные цифры $v$ могут быть вычислены путем нахождения десятичного показателя $t$, для которого $1 \leqslant \cfrac{f_v \times 10^t}{2^{-e_v}} < 10$.

Первая цифра является целой частью этой дроби. 
Последующие цифры вычисляются путем повторного использования оставшейся дроби: нужно умножить числитель на $10$ и взять целую часть от вновь полученной дроби.

Идея \textsf{Grisu} состоит в кэшировании приблизительных значений дробей $\cfrac{10^t}{2^{e_t}}$.
Дорогостоящих операций с большими числами не будет: они заменяются операциями с целыми числами фиксированного размера.

Кэш для всевозможных значений $t$ и $e_t$ может быть весьма затратным. 
Из-за этого требования к кеш-памяти в \textsf{Grisu} упрощены: кэш хранит только нормированные приближения с плавающей точкой всех соответствующих степеней десяти, а не самих дробей. Таким образом хранятся $$\tilde{c}_k := \left[ 10^k \right]_q^{\star},$$ где $q$ -- точность кэшированных чисел.
Кэшированные числа сокращают большую часть вычисления экспоненты $v$, так что остается вычислить только показатель. 

В процессе вычисления знаков используются степени десяти с экспонентой $e_{\tilde{c_t}}$, близкой к $e_v$. 
Разница между двумя показателями будет небольшой.
Фактически, \textsf{Grisu} выбирает степени десяти так, что разница лежит в определенном диапазоне. 

\paragraph{Реализация.}
Алгоритм \textsf{Grisu}: \begin{itemize}
\item \textit{Вход:} положительное число с плавающей точкой $v$ точности $p$.
\item \textit{Условие:} точность \texttt{diy\_fp} удовлетворяет $q \geqslant p + 2$, а кеш степеней десяти состоит из предварительно вычисленных нормированных округленных  \texttt{diy\_fp} со значениями $\tilde{c}_k := \left[ 10^k \right]_q^{\star}$
\item \textit{Вывод:} строковое представление в основании 10 для $V$ такое, что $[V]^{\Box}_p = v$. 
То есть при чтении числа $V$, оно должно быть округлено до $v$.
\end{itemize}

Шаги алгоритма:
\begin{enumerate}
\item \textit{Преобразование:} определим нормированный \texttt{diy\_fp} $w$ такой, что $w = v$.
\item \textit{Кэширование степеней:} сначала вычисляем $$k = -\left\lceil \log_{10} {2^{\alpha -e+q-1}} \right\rceil,$$  а затем находим с заданной точностью $$\tilde{c}_{-k} = f_c \times 2^{e_c}$$ такое, что $\alpha \leqslant e_c + e_w + q \leqslant \gamma$. ($\alpha$ и $\gamma$ заданные заранее параметры, причем $\alpha + 3 \leqslant \gamma$. Считаем $\alpha=0$ и $\gamma=3$).
\item \textit{Произведение:} пусть $$\tilde{D} = f_D \times 2^{e_D} := w \otimes \tilde{c}_{-k}.$$
\item \textit{Выход:} определим искомое $$V := \tilde{D} \times 10^k.$$ 
Вычислим десятичное представление $\tilde{D}$, за которым следует строка \texttt{e} и десятичное представление $k$.
\end{enumerate}
Поскольку значение \texttt{diy\_fp} больше, чем значение входного числа, преобразование в шаге 1 дает точный результат. 
По определению \texttt{diy\_fp}-ы имеют бесконечный диапазон экспоненциальности, и следовательно, показатель степени $w$ достаточно велик для нормирования. 
Заметим, что показатель $e_w$ удовлетворяет $e_w \leqslant e_v - (q - p)$. 
 
Легко показать, что $\forall \, i \quad 0 < \tilde{e}_{c_i} - \tilde{e}_{c_{i-1}} \leqslant 4$, и поскольку кеш неограничен, требуемый $\tilde{c}_{-k}$ должен находиться в кэше. 
Это является причиной первоначального требования $\gamma \geqslant \alpha + 3$.
Разумеется, бесконечный кеш не нужен. 

Результатом \textsf{Grisu} является строка, содержащая десятичное представление $\tilde{D}$, за которым следуют символ \texttt{e} и $k$ знаков. 
Таким образом, он представляет собой число $V: = \tilde{D} \times 10^k$. 
Утверждается, что представление $V$ у числа $v$ \textit{удовлетворяет требованию} с точностью $p$.

\subsubsection{Алгоритм \textsf{Grisu2}}
Но у \textsf{Grisu} есть существенный недостаток.
Например, при значениях по умолчанию $\alpha = 3$ и $q=64$ получаем $k=-19$, а значит, число 1 будет напечатано в виде \texttt{10000000000000000000e-19.}
Поэтому будем использовать \textsf{Grisu2}.
Этот алгоритм является усовершенствованием предыдущего и не записывает лишние нули в конец числа.
Так, если целочисленный тип \texttt{diy\_fp} содержит более двух дополнительных битов, или так называемых флагов, то эти флаги можно использовать для сокращения длины выходной строки.
В отличие от \textsf{Grisu} алгоритм \textsf{Grisu2} не генерирует полное десятичное представление, а просто возвращает значащие цифры и соответствующий показатель. 
Затем процедура форматирования объединяет эти данные для получения представления в требуемом формате.

\paragraph{Идея алгоритма.} 
Как описано выше, \textsf{Grisu2} использует дополнительные флаги для создания более короткой выходной строчки. 
Также \textsf{Grisu2} не будет работать с точными числами, а вместо этого будет вычислять аппроксимации $m^{-}$ и $m^+$. 
Чтобы избежать ошибочных результатов, которые не удовлетворяют требованиям, добавляется так называемое безопасное пространство (\textit{safety margin}) вокруг приблизительных границ.
То есть увеличивается диапазон, в котором, согласно алгоритму, может оказаться полученное число.  
Как следствие, \textsf{Grisu2} иногда может вернуть не самое оптимальное представление: оно может лежать вне изначальных границ. 
Во избежание таких проблем добавляется третий дополнительный флаг. Итого, $q \geqslant p + 3$.

\paragraph{Реализация.}
Алгоритм \textsf{Grisu2}: \begin{itemize}
\item \textit{Вход:} положительное число с плавающей точкой $v$ точности $p$.
\item \textit{Условие:} точность \texttt{diy\_fp} удовлетворяет $q \geqslant p + 3$, а кеш степеней десяти состоит из предварительно вычисленных нормированных округленных  \texttt{diy\_fp} значений $\tilde{c_k} := \left[ 10^k \right]_q^{\star}$
\item \textit{Вывод:} десятичные знаки $d_i$, где $0 \leqslant i \leqslant n$ и целочисленное $K$, такое что $V:=d_0\dots d_n \times 10 ^K$ удовлетворяет $[V]^{\Box}_p = v$. 
\end{itemize}

Шаги алгоритма:
\begin{enumerate}
\item \textit{Границы:} вычисляем границы для $v$: $m^{-}$ и $m^{+}$. 
\item \textit{Преобразование:} определим \texttt{diy\_fp} для $w^+$ так ,что $w^+ = m^+$. 
Определим также \texttt{diy\_fp} для $w^-$ так, что $w^{-} = m^-$ и $e_w^- = e_w^+$.
\item \textit{Кэширование степеней:} находим с заданной точностью $$\tilde{c}_{-k} = f_c \times 2^{e_c}$$ такое, что $\alpha \leqslant e_c + e_w + q \leqslant \gamma$.
\item \textit{Произведение:} вычисляем \begin{align*}
&\tilde{M}^- := w^- \otimes \tilde{c}_{-k}; \\
&\tilde{M}^+ := w^+ \otimes \tilde{c}_{-k};
\end{align*}и пусть также \begin{align*} & M_\uparrow^- := \tilde{M}^- + 1 \texttt{ulp};\\
& M_\downarrow^+ := \tilde{M}^+ - 1 \texttt{ulp};\\
& \delta := M_\downarrow^+ - M_\uparrow^-.
\end{align*}
\item \textit{Количество разрядов:} находим наибольшее $\kappa$ такое,\\ что $M_\downarrow^+ \mod 10^\kappa \leqslant \delta$ и определим $P:= \left\lfloor \cfrac{M_\downarrow^+}{10^\kappa} \right\rfloor$.
\item \textit{Выход:} определим $V:= P \times 10^{k + \kappa}$. 
Получаем десятичные знаки $d_i$ и число $n$ путем вычисления десятичного представления $P$.
Положим $K:=k+\kappa$ и возвращаем его с $n$ знаками $d_i$.
\end{enumerate}

\textsf{Grisu2} не дает никаких гарантий относительно краткости длины результата. 
Его результатом является кратчайшее возможное число в интервале от $M_\uparrow^- \times 10^k$ до $M_\downarrow^+ \times 10^k$ включительно, где $M_\uparrow^- \times 10^k$ и $M_\downarrow^+ \times 10^k$  зависят от точности $q$ для \texttt{diy\_fp}.
Чем больше $q$, тем ближе $M_\uparrow^- \times 10^k$ и $M_\downarrow^+ \times 10^k$ к фактическим границам $m^-$ и $m^+$. 

Для наглядности на Рисунке \ref{pic2} приведен пример работы \textsf{Grisu2}.
В частности, видно, что алгоритм отбрасывает лишние нули в конце числа и что выбирает более короткую запись числа. 
Иррациональное число \textsf{Grisu2} напечатал с машинной точностью.
\begin{figure}[h!]
\begin{center}
\texttt{ 
\begin{tabular}{|l|c|l|}
\hhline{|-|~|-|}
  array[0] = 1;				& & 1 \\
  array[1] = 1.2;			& & 1.2 \\
  array[2] = 1.23;			& & 1.23\\
  array[3] = 1.23400000;			& & 1.234 \\
  array[4] = 1.23456789;	& & 1.23456789\\
  array[5] = -1;			& & -1\\
  array[6] = -1.234;		& $\longrightarrow$ & -1.234\\
  array[7] = sqrt (2);		& & 1.4142135623730952\\
  array[8] = 1234e-36;		& & 1.234e-33\\
  array[9] = 0.000000123;	& & 1.23e-7\\
  array[10] = 0.123;		& & 0.123\\
  array[11] = 12.3;	& & 12.3\\
  array[12] = 123.000;	& & 123\\
\hhline{|-|~|-|}
\end{tabular}
}
\textit{ 
\begin{tabular}{ccc}
$\qquad\quad$ массив $\qquad\quad$ & $\qquad\qquad$ & выходной файл \\
\end{tabular}
}
\caption{Полученный с помощью \textsf{Grisu2}, выходной файл для данного массива.} \label{pic2}
\end{center}
\end{figure}

\subsubsection{Улучшения для алгоритма}
Будем использовать следующее улучшение.
Пусть в массиве есть $n$ подряд идущих одинаковых чисел с заданной точностью, то есть $\forall i: \, 0 \leqslant i \leqslant n$ верно, что $\|x_i - x_{i+1}\| \leqslant \varepsilon$, где $\varepsilon$ -- машинная точность.
В таком случае сократим запись $n$ чисел и вернем строку вида \texttt{n*x}.
Таким образом, если в нашем массиве много повторяющихся чисел, то выходная строчка будет гораздо короче, а значит, можно ускорить программу и уменьшить размер выходного файла.
Для этого также нужно модифицировать алгоритм чтения, чтобы он мог обрабатывать запись вида \texttt{n*x}.

Запись целых чисел, означающих количество повторяющихся элементов массива, также можно ускорить.
Для этого используется алгоритм 2 из статьи \cite{2}.
Суть алгоритма заключается в быстром логарифмировании числа по основанию 10.
